\documentclass[conference]{IEEEtran}
\IEEEoverridecommandlockouts
% The preceding line is only needed to identify funding in the first footnote. If that is unneeded, please comment it out.
\usepackage{cite}
\usepackage{amsmath,amssymb,amsfonts}
\usepackage{algorithmic}
\usepackage{graphicx}
\usepackage{textcomp}
\usepackage{xcolor}
\def\BibTeX{{\rm B\kern-.05em{\sc i\kern-.025em b}\kern-.08em
    T\kern-.1667em\lower.7ex\hbox{E}\kern-.125emX}}
\begin{document}

\title{Ransomware threat success factors, taxonomy, and countermeasures: A survey and research directions-Articulo1\\

\thanks{Identify applicable funding agency here. If none, delete this.}
}

\author{\IEEEauthorblockN{1\textsuperscript{st} Nicolas Romero Diaz}

}

\maketitle

\begin{abstract}

This document describes Ransomware 
for the locked data. Therefore, ransomware has become a lucrative business that has gained
increasing popularity among attackers. Unlike traditional malware, even after removal,
ransomware’s effect is irreversible and difficult to mitigate without the help of its creator.
In addition to the downtime costs and the money that individuals and business entities
threat and have tried to provide detection and prevention solutions. However, there is a lack
of survey articles that explore.

\end{abstract}

\begin{IEEEkeywords}
component, formatting, style, styling, insert
\end{IEEEkeywords}

\section{\textbf{Introduction}}

The main idea of the article is to provide a comprehensive review of the research on ransomware, including its threat success factors, taxonomy, and countermeasures. The article aims to highlight the challenges and issues faced by existing solutions to ransomware threats and to provide suggestions for future research in this area.



\section{\textbf{Descripcion del Problema}}

The problem addressed in this PDF file is the threat of ransomware, which is a type of malware that encrypts a victim's files and demands payment in exchange for the decryption key. The authors note that ransomware attacks can result in significant costs to victims, including downtime costs, ransom payments, loss of data, and damage to reputation. The authors also highlight the lack of survey articles that explore the research endeavors in ransomware and highlight the challenges and issues faced by existing solutions. The survey aims to fill this gap by providing a holistic state-of-the-art review of the research on ransomware and its detection and prevention techniques.




\section{\textbf{REFERENCIAS}}


\begin{itemize}
\item Anderson, R., & Moore, T. (2006). The Economics of Information Security. Science, 610(2006), 1–11. https://doi.org/10.1126/science.1130992
\item https://view.genial.ly/58f0b95d2655e14b541feb54/interactive-content-foros-de-discusion-caracteristicas-y-objetivos-generales
\item https://j2logo.com/tutorial-flask-leccion-5-base-de-datos-con-flask-sqlalchemy/
\item https://www.epitech-it.es/flask-python/
\end{itemize}







\end{document}
